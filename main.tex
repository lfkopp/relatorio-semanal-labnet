%%%%%%%%%%%%%%%%%%%%%%%%%%%%%%%%%%%%%%%%%%%%%%%%%%%%%%%%%%%%%%%%%%%%%%
% How to use writeLaTeX: 
%
% You edit the source code here on the left, and the preview on the
% right shows you the result within a few seconds.
%
% Bookmark this page and share the URL with your co-authors. They can
% edit at the same time!
%
% You can upload figures, bibliographies, custom classes and
% styles using the files menu.
%
%%%%%%%%%%%%%%%%%%%%%%%%%%%%%%%%%%%%%%%%%%%%%%%%%%%%%%%%%%%%%%%%%%%%%%

\documentclass[12pt]{article}

\usepackage{sbc-template}

\usepackage{graphicx,url}

%\usepackage[brazil]{babel}   
\usepackage[utf8]{inputenc}  

     
\sloppy

\title{Relatório Semanal}

\author{Nome Completo}



\begin{document} 

\maketitle

\begin{resumo} 
  Relatório da semana de DD/MM/20AA a DD/MM/20AA
\end{resumo}

 
\section{O que você realizou na semana?}
Aqui você deve inserir os resultados, papers encontrados, apresentações feitas, textos criados e afins. Refeência exemplo \cite{aquino2016hephaestus}.


\section{Como isso vai contribuir na sua tese?} 
Colocar aqui como vai contribuir.


\section{Quais os objetivos para a próxima semana?}
Colocar aqui os objetivos da proxima semana.


\section{O que você pretende me mostrar como resultado?}
Colocar aqui os objetivos mais a longo prazo.


\bibliographystyle{sbc}
\bibliography{bibliografia_semanal}

\end{document}
